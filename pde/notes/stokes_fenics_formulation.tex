\documentclass[]{article}

\usepackage[margin=1.5in]{geometry}
\usepackage{hyperref}
\usepackage{csvsimple}
\usepackage{arydshln}
\usepackage{amsmath}
\usepackage{amsfonts}
\usepackage{mathtools}
\usepackage{cleveref}
\usepackage{empheq}

\newcommand*\widefbox[1]{\fbox{\hspace{2em}#1\hspace{2em}}}


\newcommand{\ds}{\displaystyle}
\newcommand{\sm}{\sum_{n=1}^\infty}
\newcommand{\be}{\begin{enumerate}}
\newcommand{\ee}{\end{enumerate}}
\newcommand{\bi}{\begin{itemize}}
\newcommand{\ei}{\end{itemize}}
\newcommand{\GG}{\mathcal{G}}
\newcommand{\CC}{\mathbb{C}}
\newcommand{\ZZ}{\mathbb{Z}}
\newcommand{\RR}{\mathbb{R}}
\newcommand{\NN}{\mathbb{N}}
\newcommand{\qq}{\mathbb{Q}}
\newcommand{\dif}{\mathrm{d}}
\newcommand{\uu}{\mathbf{u}}
\newcommand{\vv}{\mathbf{v}}
\newcommand{\rr}{\mathbf{r}}
\newcommand{\xx}{\mathbf{x}}
\newcommand{\om}{\omega}
\newcommand{\omm}{\vec{\omega}}
\newcommand{\x}{\times}
\newcommand{\curl}{\nabla \times}
\newcommand{\dvr}{\nabla \cdot}
\newcommand{\grad}{\nabla}
\newcommand{\pt}{\partial}
\newcommand{\cross}{\times}
\newcommand{\trm}{\textrm}
\newcommand{\EE}{\mathbf{E}}
\newcommand{\BB}{\mathbf{B}}
\newcommand{\laplace}{\nabla^2}
\newcommand{\ldarrow}{\longleftrightarrow}
\newcommand{\darrow}{\leftrightarrow}
\newcommand{\eq}{\begin{equation}}
\newcommand{\qe}{\end{equation}}
\newcommand{\eqnref}[1]{eqn.~\ref{#1}}
\newcommand{\figref}[1]{Fig.~\ref{#1}}

\newcommand{\eps}{\varepsilon}
\newcommand{\ab}{\alpha \beta}
\newcommand{\ttau}{\bm{ \tau}}
\begin{document}

\title{FE Stokes equations}
\author{Noah}
\maketitle

Stokes can be written as 
\begin{align}
- \dvr (\grad \mathbf{u} + p I) &= \mathbf{f} \\
\dvr \mathbf{u} &= 0.
\end{align}

The na\"ive finite element approach to solving the Stokes equations is as follows. 
Find $(u,p) \in W$ such that
\eq
a((\mathbf{u}, p), (\mathbf{v}, q)) = L((\mathbf{v},q))
\qe
for all $(v, q) \in W$, where
\begin{align}\label{prob0a}
a ((\mathbf{u},p), (\mathbf{v},q)) &= \int_\Omega \grad \mathbf{u} \cdot \grad \mathbf{v} - \grad \cdot \mathbf{v} p + \grad \cdot \mathbf{u} q \, \dif x,
\\ 
\label{prob0b}
L((\mathbf{v},q)) &= \int_\Omega \mathbf{f} \cdot \mathbf{v} \, \dif x.
\end{align}

Using first order elements in both velocity and pressure leads to stability problems for reasons that are unclear to me but clear to some in the FEA community.
Replace equations~\ref{prob0a} and~\ref{prob0b} with 
\begin{align}
a ((\mathbf{u}, p), (\mathbf{v},q)) &= \int_\Omega \grad \mathbf{u} \cdot \grad \mathbf{v} - \grad \cdot \mathbf{v} p + \grad \cdot \mathbf{u} q  + \delta \grad q \cdot \grad p \, \dif x,
\\
L((\mathbf{v},q)) &= \int_\Omega \mathbf{f} \cdot \mathbf{v} \, \dif x 
+ \int_\Omega \grad q \cdot \mathbf{f} \, \dif x.
\end{align}
We then implement this on FEniCS with a mixed function space method.



\end{document}